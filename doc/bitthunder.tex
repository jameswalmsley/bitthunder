\documentclass[10pt,a4paper,final]{article}

\usepackage[utf8]{inputenc}
\usepackage{amsmath}
\usepackage{amsfonts}
\usepackage{amssymb}
\usepackage{textcomp}
\usepackage{listings}
\usepackage{color}
\usepackage{moreverb}
\usepackage{multicol}
\usepackage{underscore}
\ifx\pdftexversion\undefined
\usepackage[dvips]{graphicx}
\else
\usepackage[pdftex]{graphicx}
\DeclareGraphicsRule{*}{mps}{*}{}
\fi

\usepackage[cm]{fullpage}

%\usepackage{listingray}{gray}{0.9}
%\usepackage{lbcolor}{rgb}{0.9,0.9,0.9}

\begin{document}

\title{\huge{BitThunder}}
\author{James Walmsley}
\date{December 2012}
\maketitle

\begin {abstract}
BitThunder is a reliable Real-Time operating system and application framework. It provides
a complete platform abstraction layer above which embedded application software can sit. This allows
application software to be completely decoupled from hardware proprietary libraries.

In theory, such applications should require almost no porting to other platforms, provided
a complete hardware abstraction layer is implemented for the new platform.
\end {abstract}

\newpage
\tableofcontents
\newpage

\part{Overview}
%\begin{multicols}[2]
\section{Introduction}

BitThunder aims to fill the middle groud between FreeRTOS and Linux. In fact the primary
scheduler used is currently FreeRTOS, however this can easily be exchanged with a tasking
kernel of your choice. In the future we may even provide our own kernel that more closely
integrates with BitThunder.

BitThunder can be compiled as a fully-fledged RTOS, but also there is an option to compile
BitThunder as a library. This will allow Windows and Linux applications to take advantage
of some nice structures and APIs. E.g. linked lists and B+Trees etc.

BitThunder aims to remain real-time, while providing some advanced features like
process isolation, device management, and tracing etc.

\newpage
\section{Scope (v1.0.0) - End of 2013}
This is a short summary that shall describe the scope for the v1.0.0 BitThunder implementation.
Any requests/features required outside of this scope will have to wait for future BitThunder development
schedules. \textbf{ABSOLUTELY NO EXCEPTIONS}.

\subsection{Basic Platform Abstraction}
BitThunder will provide a mechanism to abstract the basic elements of an embedded micro-processor platform.
This will include:

\begin{itemize}
\item Single Global Machine Description / Entry point.
\item Interrupt Controller abstraction.
\item GPIO Controller abstraction.
\item System Timer Abstraction.
\end{itemize}

\subsection{Basic Device Management}
BitThunder will provide a simple mechanism for describing devices and their associated resources.
There will be a clear distinction between device description, and the associated driver required
to utilise the device. This will ensure BitThunder drivers are easily ported between platforms.

\subsubsection{Device Interfaces / APIs}
BitThunder's device manager can support any device interface, but only the following device types
will be supported in v1.0.0

\begin{itemize}
\item Interrupt controllers.
\item GPIO controllers.
\item UART devices.
\item Timer devices.
\item I2C Bus controllers.
\item CAN Bus controllers.
\item Ethernet / MAC interface controllers.
\item Basic block devices.
\item SDIO controllers.
\end{itemize}

\subsection{Core Operating System Services}
BitThunder will provide the following core operating system services.

\begin{itemize}
\item Create/Destroy processes.
\item Create/Destroy threads.
\item Basic memory manager (alloc/free).
\item Thread synchronisation objects (mutex/semaphores).
\item IPC (Inter-process communication) (message queues).
\item Virtual filesystem and basic APIs.
\item Manage volumes, partitions, and file-system mountpoints.
\item Open/Close devices.
\item Provide access to mounted filesystems and files.
\item Basic network stack and APIs.
\item High and low-resolution kernel timestamps.
\item System logging.
\item Process watchdog.
\end{itemize}

\subsection{The answer is NO!}
For clarity, and as reference for my refusal to compromise here is a quick list of items
that do not come under v1.0.0 scope. Thats not to say that BitThunder cannot or will not
support these features, simply they will not be planned until after completion of the 
v1.0.0 milestone.

\begin{itemize}
\item Real-time tracing.
\item MMU Support / Real process isolation.
\item USB (all versions) and other complex plug and play busses.
\item Support for high-level (libc based) applications. E.g. porting Samba / vi / ssh.
\end{itemize}




%\end{multicols}

\newpage
\part{Architecture}
\section{Sub Systems}
BitThunder can be separated into a number of sub-systems. There are some base sub-systems,
which all other sub-systems use for allocating and managing resources.

This document covers the sub-systems required for version 1.0.0 of BitThunder.

\begin{tabular}{|l|l|r|}
\hline 
Subsystem & Description & Completeness \\ 
\hline 
Handle Manager & Tracks all resources via handles, ensuring cleanup is possible. & 90\% \\ 
\hline 
Device Manager & Provides abstracted access to devices and their resources. & 50\% \\ 
\hline 
Process Manager & Fundamental process abstraction, allows clean killing. & 75\% \\ 
\hline
Real-Time Scheduler & Provides real-time slicing of the CPU resource. & 100\% \\ 
\hline 
Volume Manager & Manages volumes and their partitions, based on block device media. & 0\% \\ 
\hline
Filesystem Manager & Abstracts file-system implementations. & 50\% \\ 
\hline 
Network Stack & Manages network device and network APIs. & 25\% \\ 
\hline 
\end{tabular} 

\subsection{Handle Manager}
Provides management of process / kernel level handles. Tracks their resources, and provides a novel cleanup
mechanism.

The BT_HANDLE type is a fundamental type in BitThunder. Almost all resources have an associated BT_HANDLE in some
way. 

\subsubsection{Data Structures}
\includegraphics{figures/architecture/handles.1}


\subsection{Device Manager}
Provides several "directories" for finding and enumerating abstracted devices.
\begin{itemize}
\item BT_MACHINE_DESCRIPTION
\item BT_INTEGRATED_DEVICE
\end{itemize}

The device manager, is one of the most complex aspects of BitThunder...
\subsubsection{Machine Descriptions}
\subsubsection{Integrated Devices}
\subsubsection{GPIO}
\subsubsection{Interrupts}

\subsection{Memory Manager}
Provides APIs for allocating and freeing memory. To begin with a simple implementation is provided
but this should be replaced later, as the requirement for using protected / virtual memory becomes real.

\subsection{Process Manager}

\subsection{Real-Time Scheduler}
This is the core of the process manager, providing the fundamental time-slicing of the CPU resource.


\subsection{Block Device Manager}

\subsection{Volume Manager}

\subsection{Filesystem Manager}

\subsection{Network Stack}




\newpage
\part{API Reference}
\section{Core Operating System Services}
\subsection{Processes}

\subsection{Threads}


\end{document}

